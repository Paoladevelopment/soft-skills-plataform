 A continuación se presenta un estudio de la literatura que abarca libros y estudios relacionados con los temas principales, así como investigaciones sobre gamificación y proyectos similares. También se analizan los argumentos a favor y en contra, considerando los aspectos más relevantes para este caso de estudio en particular.
\\ \\

\begin{longtable}{|p{3cm}|p{6cm}|p{4cm}|p{4cm}|}
\hline
\textbf{Nombre}  & \textbf{Resumen} & \textbf{Pros} & \textbf{Contras} \\ 
\hline
\endfirsthead
\hline
\textbf{Nombre}  & \textbf{Resumen} & \textbf{Pros} & \textbf{Contras} \\ 
\hline
\endhead
\hline
\endfoot

Perspectivas en nutrición humana Escuela de Nutrición y Dietética
de la Universidad de Antioquia
Vol. 21, N.° 1\cite{1} & El documento ofrece una visión detallada sobre la importancia de la alimentación en la prevención de enfermedades crónicas no transmisibles, destacando la necesidad de promover el consumo diario de frutas y verduras. Asimismo, subraya la relevancia de adoptar un estilo de vida saludable, que integre la práctica regular de actividad física y la evitación de hábitos nocivos, como el consumo de tabaco y alcohol. & Una adecuada contextualización sobre las enfermedades provocadas por una alimentación deficiente, complementada con una sólida recopilación de datos.
 & Este es un documento breve de solo 6 páginas, que funciona más como una introducción que como un estudio exhaustivo.
 \\ 
\hline
Influencia del deporte y la actividad física en el estado de salud físico y mental\cite{2}
&
El artículo tiene como objetivo principal explorar y describir los beneficios que la práctica del deporte y la actividad física aportan al estado de salud, considerando tanto los aspectos físicos como mentales. Se destacan los efectos positivos de estas actividades en la salud, incluyendo la prevención de enfermedades crónicas, la mejora de la salud cardiovascular, y la reducción del estrés, la ansiedad y la depresión. Además, se subraya el fortalecimiento de habilidades cognitivas y sociales como otro de los beneficios significativos de la actividad deportiva. & El artículo presenta una revisión bibliográfica exhaustiva sobre la influencia del deporte y la actividad física en la salud física y mental, ofreciendo una visión clara de los beneficios asociados a la actividad física. Además, incluye una amplia variedad de fuentes confiables que respaldan sus hallazgos.
 & Dado que se basa en una revisión bibliográfica, el artículo incluye una gran cantidad de datos; sin embargo, no profundiza de manera extensiva en ningún tema específico.
 \\ 
\hline

Estrategias Gamificación aplicadas a la Educación y a la Salud\cite{3}
&
Este documento proporciona una exploración detallada de la integración de la gamificación en el ámbito educativo, con un enfoque particular en el diseño de interfaces interactivas. Se destaca la aplicación de elementos tanto individuales como sociales, con el objetivo de motivar a los estudiantes y potenciar su proceso de aprendizaje. Además, se centra en TANGO:H, una tecnología basada en el uso de Kinect de Microsoft©, que, a través de una cámara RGB y un sensor de profundidad, es capaz de reconocer el cuerpo humano.
 & La propuesta del artículo se diseñó y verificó a través de una clase de jóvenes, destacando las características de gamificación empleadas en el proceso.&El documento está demasiado centrado en TANGO:H
 \\ 
\hline
Gamificación en Educación Física\cite{4}
&
El documento resalta la importancia de implementar la gamificación en la educación física como una estrategia metodológica para enriquecer el proceso formativo de los estudiantes. Se subraya la necesidad de proporcionar experiencias motrices auténticas y significativas, alineadas con las necesidades actuales de los jóvenes, con el objetivo de fomentar un repertorio motriz amplio y diverso que tenga un impacto positivo en la salud. Además, el análisis se apoya en la revisión de 26 fuentes diferentes, lo que le otorga una sólida base teórica.
 & El documento demuestra el crecimiento significativo de la gamificación en diversos ámbitos educativos, destacando cómo esta tendencia ha transformado las prácticas pedagógicas. Se promueve una experiencia de aprendizaje que enriquece y fortalece el proceso formativo de los estudiantes, motivándolos a participar activamente y mejorando tanto su compromiso como su rendimiento académico.&Aunque es un buen recopilatorio, la información se limita a verificar el crecimiento del uso de la gamificación, lo que reduce su relevancia para este caso de estudio en particular. 
 \\ 
\hline
Play the Game: gamificación y hábitos saludables en educación física \cite{5}
&
El documento se enfoca en la implementación de una unidad didáctica gamificada denominada "Play The Game", llevada a cabo en centros educativos de Barcelona durante el primer trimestre del curso académico 2013-2014. Esta iniciativa se estructuró en torno a unidades didácticas diseñadas para fomentar la interacción entre el alumnado. La unidad comprendió un total de 12 sesiones de una hora cada una, y la muestra del estudio incluyó 3 profesores y 9 alumnos seleccionados a partir de proyectos previos.
 & El documento presenta 7 niveles de retos o juegos diseñados para el aprendizaje del alumnado, utilizando herramientas que abarcan desde formatos presenciales hasta virtuales. Además, se destacan metodologías efectivas que promueven el desarrollo individual de los estudiantes, potenciando tanto sus habilidades académicas como personales a través de la gamificación.
&Los datos analizados provienen de estudiantes preseleccionados, lo que podría influir en los resultados y limitar la generalización de los hallazgos. Además, el artículo tiene 9 años de antigüedad, lo que podría afectar la relevancia de sus conclusiones en el contexto educativo actual, dado que la tecnología y las metodologías han evolucionado considerablemente desde entonces.
 \\ 
\hline
How to win friends and influence people\cite{6}
&
El libro se enfoca en técnicas esenciales para interactuar de manera efectiva con los demás. Dale Carnegie subraya la importancia de demostrar un interés genuino en las personas, practicar una escucha activa y ofrecer elogios sinceros. También propone estrategias clave para ganar la simpatía de los demás, tales como recordar y utilizar el nombre de la persona, conversar en función de sus intereses y evitar hacer críticas directas. Estas técnicas buscan mejorar las relaciones interpersonales y fomentar una comunicación más efectiva y positiva.
 & El libro ofrece diversas pautas sobre cómo influir en otras personas mediante el autodesarrollo y la influencia en uno mismo. Destaca la importancia de mejorar nuestras propias habilidades interpersonales y actitudes, como clave para impactar positivamente en los demás. A través de la autogestión y el control emocional, el autor sugiere que es posible ejercer una influencia más efectiva sobre quienes nos rodean.&La primera mitad del libro se enfoca en las interacciones con otras personas, mientras que la segunda mitad resulta ser la más útil para este caso de estudio. 
 \\ 
\hline
Think and Grow Rich \cite{7}
&
"Piense y Hágase Rico" de Napoleón Hill es una obra que examina cómo los pensamientos influyen en la creación de riqueza y éxito personal. El autor destaca el poder de la autosugestión, la fe y una planificación meticulosa como elementos clave para alcanzar el éxito. Asimismo, presenta diversas herramientas y estrategias diseñadas para convertir los pensamientos en acciones concretas que faciliten el logro de las metas propuestas.
 & El libro ofrece valiosos consejos enfocados en la autosuperación y la motivación personal, proporcionando estrategias prácticas para desarrollar una mentalidad positiva y proactiva.&El libro no se basa en fundamentos científicos, sino que recurre a enfoques empíricos, apoyándose en experiencias personales y ejemplos anecdóticos para sustentar sus ideas. 
 \\ 
\hline
Psycho-Cybernetics \cite{8}
&
"Psycho-Cybernetics", es un libro de autoayuda escrito por Maxwell Maltz en 1960, que emplea aspectos psicológicos del ser humano para promover el mejoramiento de la autoimagen. Maltz sostiene que una imagen propia positiva es clave para alcanzar una vida más exitosa y satisfactoria, ofreciendo herramientas y estrategias para reprogramar la mente y modificar patrones de pensamiento que impiden el crecimiento personal.
 & El libro fue avalado y ampliamente elogiado por diversos autores, consolidando su reputación como una obra influyente en el ámbito del desarrollo personal. Además, incluye múltiples anécdotas que, junto con las actualizaciones presentes en la versión de 2002, lo convierten en un libro completo y enriquecedor, proporcionando una visión integral del proceso de mejora de la autoimagen.&Aunque es un libro relativamente antiguo, su enfoque en la autopercepción lo mantiene plenamente vigente en nuestro contexto actual. Los principios que aborda, relacionados con la autoimagen y el desarrollo personal, siguen siendo relevantes y aplicables, independientemente del paso del tiempo. 
 \\ 
\hline
Atlas Shrugged \cite{9}
&
La rebelión de Atlas es una novela que narra una rebelión ficticia de grandes empresarios contra los políticos y el gobierno de los Estados Unidos. La obra retrata a los empresarios vanguardistas como el "Atlas" (el dios griego que sostiene el mundo), simbolizando su carga de responsabilidad y sus luchas en una sociedad que depende de ellos desesperadamente, pero que al mismo tiempo los oprime y desprecia. La novela explora los desafíos y sacrificios que enfrentan en un mundo que necesita su innovación y liderazgo para prosperar.
 & La novela fomenta eficazmente una actitud positiva y la autosugestión, destacando la importancia de la determinación y el poder personal frente a las adversidades.&La novela es considerablemente extensa, con un total de 1,058 páginas, lo que la convierte en una lectura densa. Además, cuenta con una antigüedad significativa, lo que refleja su contexto histórico y filosófico. 
 \\ 
\hline
Del gimnasio al ocio-salud \cite{10}
&
El documento ofrece información valiosa sobre la relación entre el ocio, la gestión deportiva y el turismo, subrayando la importancia de desarrollar experiencias emocionantes y satisfactorias para los consumidores. También destaca la evolución de los centros dedicados a la actividad deportiva y a la salud, evidenciando cómo han adaptado sus servicios para satisfacer las necesidades y expectativas de los usuarios por el paso del tiempo.
 & 
Proporciona información relevante sobre los centros deportivos y sus transformaciones a lo largo de la historia, contextualizando adecuadamente su importancia en el ocio y la cultura general. &El artículo fue publicado en 2007 y, aunque busca contextualizar los centros deportivos hasta la actualidad, es probable que se quede algo corto en su análisis. 
 \\ 
\hline
Desarrollo de Prototipo de sistema de recomendación para rutina de ejercicio cardiovascular\cite{11}
&
El trabajo de grado de Paola Andrea Lenis Franco se enfoca en el desarrollo de un prototipo de sistema de recomendación para rutinas de ejercicio cardiovascular. Este sistema, diseñado a través de plataformas web, permite a los usuarios registrar información sobre enfermedades cardiovasculares, lo que facilita la creación de rutinas personalizadas y seguras.
 & 
El prototipo permite la adición de nuevas enfermedades en la plataforma web, lo que asegura su actualización continua y relevancia a lo largo del tiempo. &Es necesario reevaluar la base de datos y el código de manera manual al añadir nuevas enfermedades si se desean incluir rutinas específicas para estas.  
 \\ 
\hline
Prototipo de sistema de recomendación para el apoyo a futbolistas amateur en el desarrollo de su entrenamiento deportivo de manera individual\cite{12}
&
El trabajo se centra en el desarrollo de una aplicación para un sistema de recomendación de entrenamiento deportivo individualizado, que se fundamenta en el estado físico y la capacidad deportiva de cada persona.
 & 
La usabilidad de la aplicación fue evaluada como excelente, lo que indica una experiencia positiva para los usuarios. Los inconvenientes identificados durante el primer despliegue del prototipo fueron cuidadosamente abordados y corregidos. &La aplicación tuvo reseñas mixtas pero es normal puesto que es un prototipo.  
 \\ 
\hline
Desarrollo de prototipo de aplicación móvil para entrenamientos fisicos\cite{13}
&
El trabajo se centra en el desarrollo de una aplicacion  para recomendación de entrenamientos fisicos mediante la calificacion de ciertos ejercicios por el usuario, creando a su vez un entorno mas comodo y perzonalizado para cada usuario
 & 
La aplicación incluye información relevante sobre el módulo de recomendación, destacando el uso de modelos de clasificación como One Hot Encoding y Label Encoding.Los resultados obtenidos de la aplicación cumplieron con las expectativas, lo que sugiere que el sistema de recomendación es eficiente y útil para los usuarios en la personalización de sus entrenamientos deportivos. & 
\\
\hline
Meditaciones de marco aurelio\cite{14}
&
El libro presenta las diversas meditaciones del emperador romano Marco Aurelio, en las que ofrece consejos sobre el ser, fundamentados principalmente en principios estoicos. Estas reflexiones están organizadas en 12 libros breves, cada uno de los cuales aborda diferentes aspectos de la vida, la virtud y la autoconciencia.
 & 
El libro aborda de manera sobresaliente cómo debe ser la conducta ante diversas situaciones, principalmente en la vida cotidiana. La información está estructurada en secciones, que a su vez se dividen en consejos simples y complejos, lo que permite que cada punto sea relevante por sí mismo, sin necesidad de consultar el resto del contenido. & Dado que se trata de una traducción, es recomendable releer el texto para asegurar una correcta comprensión de los conceptos y matices presentados. 
\\
\hline
Sindrome del tunel carpiano\cite{15}
&
La revisión bibliográfica ofrece un análisis introductorio sobre el síndrome del túnel carpiano, incluyendo diversos aspectos como su diagnóstico y tratamientos disponibles. Además, se proporciona información relevante que permite comprender mejor esta condición, sus síntomas y la efectividad de las intervenciones propuestas.
 & 
El documento es una excelente fuente para contextualizarse sobre el síndrome del túnel carpiano, proporcionando datos relevantes y útiles que ayudan a entender mejor esta condición y sus implicaciones. & 
\\
\hline
Terapia manual en el síndrome del túnel carpiano.\cite{16}
&
El documento es una revisión que evalúa la efectividad de la terapia manual, centrándose en las técnicas de neurodinamia y masoterapia, en el manejo clínico del síndrome del túnel carpiano (STC).
 & 
El documento detalla las técnicas e instrumentos utilizados en la evaluación, así como la catalogación de sus resultados. Se encontraron resultados satisfactorios tanto al emplear las técnicas de masoterapia y neurodinamia de manera individual como al combinarlas, lo que sugiere que estas intervenciones pueden ser efectivas en el tratamiento del síndrome del túnel carpiano. & Se requieren estudios adicionales sobre el síndrome del túnel carpiano (STC) en sus formas moderada y severa, con un mayor número de pacientes, para validar los resultados obtenidos y establecer conclusiones más robustas sobre la efectividad de las intervenciones terapéuticas.
\\
\hline
Efectividad de la movilización neurodinámica en el dolor y funcionalidad en sujetos con síndrome del túnel carpiano: revisión sistemática.\cite{17}
&
El documento es una revisión de literatura que examina la efectividad de la técnica de movilización neurodinámica en pacientes diagnosticados con síndrome del túnel carpiano.
 & 
De los cuatro estudios seleccionados, que inicialmente sumaban 63, se contó con la participación de 261 pacientes, lo que proporciona una muestra significativa para la obtención de resultados. El estudio concluyó con evidencia notable y de nivel moderado sobre la efectividad del método. Este enfoque se diseñó considerando tanto los costos como las preferencias del paciente; al tratarse de un grupo de técnicas, puede ofrecer una solución eficaz a un costo prácticamente nulo. & A diferencia de las herramientas o procedimientos quirúrgicos, las técnicas mencionadas pueden no tener el mismo impacto, ya que su efectividad puede depender de la disposición y el compromiso del paciente.
\\
\hline
Videojuegos, práctica de actividad física, obesidad y hábitos sedentarios en escolares de entre 10 y 12 años de la provincia de Granada.\cite{18}
&
La investigación examina el sobrepeso y la actividad física en una muestra de 261 niños y niñas de entre 10 y 12 años. Además, se analiza la influencia de ciertos hábitos, incluyendo el uso de videojuegos, en el estado de salud de los participantes.
 & 
El documento presenta de manera clara y efectiva los resultados obtenidos en los diferentes tipos de estudios.
 & Las conclusiones del estudio resultaron ser contrarias al objetivo inicial, evidenciando que el 55,6 por ciento de los estudiantes se encontraba en un estado de bajo peso, un hallazgo que se destacó adecuadamente en la sección de conclusiones.
\\
\hline
Motivación del estudiante y los entornos virtuales de aprendizaje (Educación a distancia y ruralidad).\cite{19}
&
El documento se centra en un análisis de las diversas plataformas educativas, evaluando su utilidad y eficacia en relación con los estudiantes.
 & 
El análisis proporciona una excelente contextualización sobre el tema, destacando los posibles problemas asociados con el uso de estas plataformas y ofreciendo orientación sobre cómo utilizarlas de manera efectiva.
 & Dado que se trata de un análisis, la información proporcionada puede ser limitada, lo que podría hacer que resulte insuficiente como material final para este proyecto.
\\
\hline
Experiencias de gamificación en aulas.\cite{20}
&
El libro "Experiencias de gamificación en aulas"  como su título indica, presenta diversas experiencias de educadores en la implementación de la gamificación, así como las estrategias empleadas por ellos. Cabe destacar que esta es la segunda edición del libro.

 & 
El libro ofrece una sólida contextualización del tema, junto con una descripción detallada de siete juegos específicos y las estrategias necesarias para implementarlos de manera efectiva en el aula.
 & 
\\
\hline
Gamificación Como Estrategia De Motivación En La Plataforma Virtual De La Educación Superior Presencial.\cite{21}
&
El documento tiene como objetivo desarrollar una propuesta de gamificación mediante la recolección de datos a través de encuestas. Además, busca proporcionar información relevante que sirva para futuras aplicaciones de la gamificación en entornos educativos.
 & 
El documento recopila y analiza una extensa cantidad de información que, independientemente de su aplicación en gamificación, puede resultar de gran utilidad para diversos propósitos.
 & El documento no pone en práctica el modelo de gamificación, lo que sugiere la necesidad de un estudio futuro para validar los resultados y la efectividad de la propuesta.
\\
\hline
La plataforma kahoot influye en la motivación durante la evaluación en los estudiantes de cuarto grado de primaria de la institución educativa nueva juventud de santa rita de siguas – Arequipa, 2020.\cite{22}
&
El documento se centra en un estudio que aborda la plataforma Kahoot desde dos enfoques. El primero es un análisis conceptual que examina la plataforma y su uso como herramienta de gamificación. El segundo enfoque se basa en una investigación sobre la opinión de los educandos que utilizaron Kahoot, a través de encuestas, para comprender su percepción de la plataforma.
 & 
El documento proporciona una visión clara de la percepción que tienen los educandos sobre una herramienta ampliamente reconocida en el ámbito de la educación gamificada. 
 & Dado que los participantes del estudio son educandos de poca edad, es posible que sus opiniones estén sesgadas a favor de la herramienta. 
\\
\hline
La resiliencia en la educación, la escuela y la vida.\cite{23}
&
El documento habla de la resilencia, utilidad e importancia en la eduacion y vida dia dia de las personas, ademas de ciertas caracteriticas que influyen en su desarrollo.
 & 
El documento, aunque es breve, es conciso y explica de manera efectiva el concepto de resiliencia y su relevancia. Proporciona una visión clara de cómo la resiliencia puede influir positivamente en la vida de las personas, especialmente en el contexto educativo.
 & El documento funciona más como un análisis introductorio sobre la resiliencia que como un desarrollo profundo del tema.Por lo tanto, podría resultar ineficaz para el contexto y los objetivos de este proyecto. 
\\
\hline
Resiliencia: Impacto Positivo En La Salud Física Y Mental\cite{24}
&
El documento analiza el impacto positivo de la resiliencia en la salud física y mental, destacando cómo la capacidad de adaptarse a situaciones adversas contribuye al bienestar general. Se enfatiza que desarrollar habilidades resilientes puede reducir problemas de salud mental y físico.
 & 
Es un buen documento para contextualizar la resiliencia y sus impactos en las personas, proporcionando información valiosa sobre su importancia en el bienestar general.
 &  
\\
\hline
La Sociedad del Espectáculo\cite{25}
&
Se trata de un libro filosófico-político que aborda el concepto del espectáculo no solo como entretenimiento, sino como una simulación de la realidad objetiva. El autor se refiere al espectáculo como la interacción entre dos personas a través de una imagen, en lugar de una conexión directa con la realidad, enfatizando la apariencia como un sustituto de la verdad.
 & 
El libro ofrece una perspectiva alternativa sobre la realidad y propone un cambio de paradigma en la forma de percibir el mundo. A través de su análisis, busca inspirar a los lectores a cuestionar las apariencias y a reflexionar sobre la autenticidad de sus experiencias.
 &  
\\
\hline
Soft Skills: The software developer's life manual\cite{26}
&
El libro "Soft Skills" es un manual dirigido a desarrolladores de software que ofrece técnicas y prácticas centradas en el desarrollo de habilidades blandas. Abarca temas como relaciones interpersonales, gestión financiera y aspectos relevantes para el proyecto, como el mindset y el fitness.
 & 
El libro contiene información importante tanto de fitness como de mindset, Además de  contar con diferentes ejemplos a tomar en cuenta sobre las habilidades blandas

 &  
\\
\hline
\caption{Tabla de estudios y proyectos relacionados}
\label{tab:estudios}
\end{longtable}


Para el desarrollo de este proyecto, se seleccionaron los siguientes trabajos:  
``Desarrollo de prototipo de aplicación móvil para entrenamientos físicos``,  
``Meditaciones de Marco Aurelio``,  
``Experiencias de gamificación en aulas``,  
``Soft Skills: The Software Developer's Life Manual`` y  
``La Sociedad del Espectáculo``.  
\\ \\
La selección se realizó considerando los pros y contras descritos en la tabla anterior, así como la estructura requerida para obtener información relevante sobre la gestión de proyectos que incorporan gamificación.
\\ \\
Además, también se tomaron en cuenta factores de mayor utilidad para este caso de estudio. Por ejemplo,  
``Desarrollo de prototipo de aplicación móvil para entrenamientos físicos`` y  
``Experiencias de gamificación en aulas`` presentaron ejemplos de gamificación.  
El primero menciona métodos para animar a un usuario desde el punto de vista del mindset, mientras que el segundo se enfoca en mejorar un aspecto físico como el fitness. Ambos trabajos presentaron resultados satisfactorios en el contexto de la gamificación, haciéndolos muy útiles por su experiencia para este proyecto.
\\ \\
En el caso de  
``Meditaciones de Marco Aurelio`` y  
``La Sociedad del Espectáculo``,  
ambos libros, a través de la experiencia de sus autores, analizan el *psique* humano y cómo, al cambiar el método de pensamiento, se puede afectar el entorno que rodea al individuo.  
El primero plantea cómo actuar para cambiar el mundo, mientras que el segundo propone modificar el análisis para verlo desde una perspectiva diferente.
\\ \\
Por último, el libro ``Soft Skills: The Software Developer’s Life Manual`` se considera la principal fuente de información sobre las habilidades blandas, ya que se enfoca en diversas experiencias tanto académicas como laborales, con gran utilidad para este caso de estudio.
\\ \\
Posteriormente, se decidió realizar un análisis detallado para determinar cómo abordar las habilidades blandas a través de la gamificación. En el caso de la categoría de fitness, se optó por enfocarse en dos problemas principales que afectan con frecuencia a los ingenieros en sistemas: la prevención del síndrome del túnel carpiano y la mejora de la postura, especialmente en áreas como el coxis, que suelen causar molestias debido a largas horas frente al ordenador.
\\ \\
Para la categoría de mindset, se decidió centrarse en algunas de las habilidades clave para los ingenieros en sistemas, muchas de ellas mencionadas en el libro "Soft Skills: The Software Developer's Life Manual" y otras identificadas como importantes para las personas en general, como en "Meditaciones de Marco Aurelio". Las habilidades seleccionadas incluyen la resiliencia, el pensamiento lógico y el aprendizaje continuo . A partir de estas, se realizó un análisis específico para determinar cómo abordarlas de manera efectiva. Cada habilidad fue subdividida según los desafíos comunes que enfrenta un ingeniero en sistemas, ya sea en su contratación o en su vida académica, y las competencias necesarias para superar dichos obstáculos.
\\ \\
A continuación, se presentan las tablas que detallan este análisis:


\begin{longtable}{|p{6cm}|p{3cm}|p{8cm}|}
\hline
\textbf{Problema}  & \textbf{Habilidad} & \textbf{Significado}  \\ 
\hline
\endfirsthead
\hline
\textbf{Problema}  & \textbf{Habilidad} & \textbf{Significado}  \\ 
\hline
\endhead
\hline
\endfoot

Dificultades al autoevaluarse como candidato potencial para una empresa & Autoconsciencia & Se refiere a la capacidad de una persona para estar consciente de sí misma,en términos de su propia existencia como de sus pensamientos, emociones, características y capacidades.

 \\ 
 \hline
Mala toma de decisiones bajo presión
&
Autocontrol
 & 
El autocontrol se refiere a la capacidad de una persona para regular y gestionar sus propias emociones, pensamientos y comportamientos en situaciones diversas. 

\\ 
 \hline
Tendencia a tener un ambiente negativo
&
Optimismo
 & 
Es una actitud mental positiva que implica esperar lo mejor en cualquier situación. 

\\ 
\hline
Las tareas tienden a ser muy diversas entre si
&
Adaptabilidad
 & 
La adaptabilidad se refiere a la capacidad de una persona para ajustarse y prosperar en entornos cambiantes o situaciones nuevas. Implica ser flexible, receptivo y capaz de cambiar de enfoque o estrategia según sea necesario en el caso de estudio o programa
\\ 
\hline

Se tiende a tener un pensamiento muy centrado en las normas por lo que ciertos problemas no tienen solución

&
Creatividad
 & 
La creatividad es la capacidad de generar ideas originales y únicas, así como de encontrar soluciones innovadoras a problemas o situaciones.


\\
\hline
Se tiende a tener una gran carga de trabajo al no completar un objetivo aunque sea en equipo
&
Gestión del estrés
 & 
La gestión del estrés se refiere a las estrategias y técnicas utilizadas para manejar y reducir los niveles de estrés en la vida diaria.

\\
\hline
Al generar una serie de soluciones no se pueden a aplicar todas

&
 Toma de decisiones & 
La toma de decisiones es el proceso mediante el cual una persona elige entre diferentes opciones disponibles con el fin de resolver un problema o alcanzar un objetivo .

\\
\hline
Se tiende a trabajar con diferentes tipos de codigo
&
Flexibilidad cognitiva
 & 
La flexibilidad cognitiva es la capacidad de adaptar y cambiar los procesos mentales y cognitivos en respuesta a nuevas situaciones, desafíos o información.
\\
\hline
El código tiende a fallar y genera estrés

&
Persistencia & 
La persistencia se refiere a la capacidad de una persona para continuar esforzándose y trabajando hacia un objetivo a pesar de los desafíos, obstáculos o fracasos que puedan surgir en el camino.

\\
\hline
Comúnmente se necesita cambiar de empleo

&
Habilidades sociales

 & 
Las habilidades sociales se refieren a las capacidades que una persona posee para interactuar de manera efectiva y apropiada con los demás en diversos contextos sociales.

\\
\hline
 Se tiende a menospreciarse a sí mismo como programador por ciertos puestos de trabajo sin experiencia

&
Autoaceptación
 & 
La autoaceptación es la capacidad de aceptarse a uno mismo tal como uno es, con todas las fortalezas, debilidades, imperfecciones y características únicas. 



\\
\hline

\caption{Tabla de subhabilidades sobre la reciliencia}
\label{tab:estudios}
\end{longtable}


\begin{longtable}{|p{6cm}|p{3cm}|p{8cm}|}
\hline
\textbf{Problema}  & \textbf{Habilidad} & \textbf{Significado}  \\ 
\hline
\endfirsthead
\hline
\textbf{Problema}  & \textbf{Habilidad} & \textbf{Significado}  \\ 
\hline
\endhead
\hline
\endfoot

Se tiene un volumen muy alto de datos.&  Análisis de datos & El análisis de datos es el proceso de examinar, limpiar, transformar y modelar datos con el objetivo de descubrir patrones, tendencias, relaciones o insights significativos que puedan utilizarse para tomar decisiones informadas o realizar predicciones. 

 \\ 
 \hline
Los datos parecen tener similitud
&
Interpretación de patrones
 & 
La interpretación de patrones se refiere al proceso de analizar datos, observar regularidades o tendencias recurrentes y extraer significado o conocimiento de ellos.

\\ 
 \hline
El programa es muy complejo y nesecita simplificarse &
Abstracción
 & 
La abstracción es un concepto que se refiere a la capacidad de representar o entender un objeto, idea o concepto de una manera simplificada, general o más abstracta, que elimina detalles innecesarios o irrelevantes. 

\\ 
\hline
Problemas matemáticos
&
Resolución de problemas lógicos
 & 
La resolución de problemas lógicos implica la aplicación de principios y reglas de la lógica para encontrar soluciones a situaciones que requieren razonamiento y análisis.
\\ 
\hline

Se encontró un problema pero muchas posibles causas

&
Pensamiento crítico & 
El pensamiento crítico es una habilidad mental que implica analizar de manera objetiva y racional la información, evidencia o argumentos presentados, con el fin de llegar a conclusiones fundamentadas y tomar decisiones informadas. 


\\
\hline
El cliente pidió un programa de forma muy abierta


&
Pensamiento analítico & 
El pensamiento analítico es una habilidad cognitiva que implica descomponer información compleja en partes más pequeñas y comprensibles para comprender mejor su estructura, relaciones y componentes individuales.
\\
\hline
Se solicita la evaluación de un programa mediante un problema


&
  Pensamiento sistemático & 
El pensamiento sistemático es una habilidad cognitiva que implica abordar problemas o situaciones de manera metódica y estructurada, siguiendo un enfoque ordenado y lógico. 

\\
\hline
El cliente pidió un programa de forma muy completa pero solicitó extras

&
Inducción y su razonamiento & 
La inducción es un proceso de razonamiento en el cual se llega a una conclusión general a partir de observaciones específicas o evidencias particulares. 

\\
\hline

\caption{Tabla de subhabilidades sobre el  pensamiento lógico}
\label{tab:estudios}
\end{longtable}


\begin{longtable}{|p{6cm}|p{3cm}|p{8cm}|}
\hline
\textbf{Problema}  & \textbf{Habilidad} & \textbf{Significado}  \\ 
\hline
\endfirsthead
\hline
\textbf{Problema}  & \textbf{Habilidad} & \textbf{Significado}  \\ 
\hline
\endhead
\hline
\endfoot

Se solicita que el empleado se adecue a un programa &  Curiosidad&La curiosidad es un rasgo fundamental del ser humano que impulsa el deseo de explorar, descubrir y aprender sobre el mundo que nos rodea.

 \\ 
 \hline
Se necesita un mayor sueldo que requiere mayor educación

&
Búsqueda de conocimiento
 & 
La búsqueda de conocimiento es el proceso activo y continuo de adquirir información, comprender conceptos y explorar nuevas ideas con el objetivo de ampliar la comprensión del mundo y enriquecer el pensamiento.

\\ 
 \hline
Los lenguajes de programación tienden a cambiar rápidamente

 &
 Autodidactismo
 & 
El autodidactismo es un enfoque de aprendizaje en el que una persona adquiere conocimientos y habilidades de manera independiente, sin la guía directa de un maestro o instructor formal.



\\ 
\hline
Se requiere adecuarse prontamente a un trabajo

&
Experimentación, Reflexión y Retroalimentación

 & 
La experimentación proporciona experiencias concretas que pueden ser reflexionadas y analizadas para extraer lecciones y aprendizajes significativos. La retroalimentación proporciona información adicional que ayuda a enriquecer la reflexión y a guiar futuras acciones y experimentos.

\\ 
\hline


\caption{Tabla de subhabilidades sobre el  aprendizaje continuo}
\label{tab:estudios}
\end{longtable}

Después de analizar las tablas, se destacaron nueve sub-habilidades, divididas en tres para cada área. En el caso de la resiliencia, las sub-habilidades son: autocontrol, adaptabilidad y toma de decisiones. Para el pensamiento lógico, se identificaron la interpretación de patrones, la abstracción y el pensamiento crítico. Finalmente, en el aprendizaje continuo se resaltaron: curiosidad, búsqueda de conocimiento y autodidactismo.
\\
En el caso de autocontrol y toma de decisiones, estas sub-habilidades fueron seleccionadas porque son cruciales para ayudar a seleccionar soluciones óptimas en tiempo limitado y bajo presión. Por otro lado, la adaptabilidad fue elegida debido a que se considera una habilidad indispensable para cualquier ingeniero en sistemas, independientemente de su área de vocación.
\\
La interpretación de patrones, la abstracción y el pensamiento crítico representan, probablemente, de la mejor forma lo que significa ser un ingeniero en sistemas capaz. Por ello, su inclusión en los ámbitos de este proyecto se considera fundamental.
\\
Por último, la curiosidad, la búsqueda de conocimiento y el autodidactismo representan los aspectos más esenciales que debe tener un estudiante, no solo para su desarrollo durante el periodo universitario, sino también para enfrentarse con éxito al mundo laboral.






