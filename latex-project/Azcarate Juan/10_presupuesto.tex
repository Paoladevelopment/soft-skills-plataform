\section{Presupuesto general}
\subsection{Gastos salariales}

La tabla en esta sección contiene los datos relevantes acerca del tiempo dedicado al proyecto, siguiendo las pautas establecidas por la según el artículo 11 del decreto 1295 del Ministerio de Educación MEN (2010) ”Un crédito académico equivale a cuarenta y ocho (48) horas de trabajo académico del estudiante, que comprende las horas con acompañamiento directo del docente y las horas de trabajo independiente que el estudiante debe dedicar a la realización de actividades de estudio, prácticas u otras que sean necesarias para alcanzar las metas de aprendizaje” y el artículo 12 del mismo decreto “teniendo en cuenta que una (1) hora con acompañamiento directo de docente supone dos (2) horas adicionales de trabajo independiente en programas de pregrado”.


\begin{table}[h]
    \centering
    
    \label{tab:resumen}
    \begin{tabular}{|l|r|}
        \hline
        \textbf{Número de créditos} & 9 \\
        \hline
        \textbf{Horas por crédito} & 48 \\
        \hline
        \textbf{Total de horas del trabajo investigativo} & 432 \\
        \hline
        \textbf{Horas disponibles del director} & 44 \\
        \hline
        \textbf{Horas totales del estudiante} & 388 \\
        \hline
    \end{tabular}
    \caption{Resumen de horas y créditos}
\end{table}

A continuación, la tabla presenta el cálculo del salario tanto para el profesor asociado como para el estudiante, basado en la tabla que detalla las horas correspondientes a cada uno:



\begin{table}[h]
    \centering
    \label{tab:costos}
    \begin{tabular}{|l|r|r|}
        \hline
        \textbf{Personal} & \textbf{Costo por hora} (pesos colombianos) & \textbf{Total} (pesos colombianos) \\
        \hline
        Estudiante & 7.250 & 2.813.000 \\
        Director trabajo de grado & 61.264 & 2.695.616 \\
        \hline
        \textbf{Total} & & 5.013.000 \\
        \hline
    \end{tabular}
    \caption{Costos salariales}
\end{table}
\newpage
\subsection{Gastos tecnológicos}
La depreciación del equipo de cómputo se calculará utilizando el método de depreciación en línea recta, considerando una vida útil de 60 meses. Este cálculo se basa en el valor inicial del equipo y su tiempo de vida útil para determinar la devaluación mensual del equipo. De esta manera, se obtiene un valor total para la duración del proyecto, que es de 8 meses


\begin{table}[h]
    \centering

    \begin{tabular}{|l|r|}
        \hline
        \textbf{Costo equipo de computo} & 1.500.000 \\
        \hline
        \textbf{Vida útil en meses} & 60 \\
        \hline
        \textbf{Desvalorización} & 25.000 \\
        \hline
        \textbf{Duración del proyecto en meses} & 8 \\
        \hline
        \textbf{Total (pesos colombianos)} & 200.000 \\
        \hline
    \end{tabular}
    \caption{Depreciación}
\end{table}

En la tabla siguiente se detalla el cálculo tanto para el costo del servicio eléctrico como para el internet durante el período necesario para el proyecto

\begin{table}[h]
    \centering
    \label{tab:detalles-proyecto}
    \begin{tabular}{|l|r|r|}
        \hline
        \textbf{Concepto} & \textbf{Costo mes} & \textbf{Valor(pesos colombianos)} \\
        \hline
        Internet & 65.000 & 520.000 \\
        Electricidad & 55.000 & 440.000 \\
        \hline
        \textbf{Total} & & 960.000 \\
        \hline
    \end{tabular}
    \caption{Detalles gasto electricidad e internet}
\end{table}

Proseguimos con la tabla del total de gastos tecnológicos:

\begin{table}[h]
    \centering
    
    \label{tab:gastos-tecnologicos}
    \begin{tabular}{|l|r|}
        \hline
        \textbf{Concepto} & \textbf{Valor (pesos colombianos)} \\
        \hline
        Equipo de computo & 960.000 \\
        Otros gastos tecnológicos & 266.667 \\
        \hline
        \textbf{Total} & 1.226.667 \\
        \hline
    \end{tabular}
    \caption{Detalles de gastos tecnológicos}
\end{table}
\newpage
\subsection{Presupuesto general total}
La tabla que sigue presenta el resumen del presupuesto total destinado al proyecto. Este presupuesto abarca los costos tecnológicos del hardware y los gastos de nómina del personal.


\begin{table}[h]
    \centering

    \label{tab:gastos}
    \begin{tabular}{|l|r|}
        \hline
        \textbf{Concepto} & \textbf{Valor (pesos colombianos)} \\
        \hline
        Gastos salariales & 5.508.616 \\
        Gastos tecnológicos & 1.226.667 \\
        \hline
        \textbf{Total} & 6.735.283 \\
        \hline
    \end{tabular}
    \caption{Gastos totales}
\end{table}

