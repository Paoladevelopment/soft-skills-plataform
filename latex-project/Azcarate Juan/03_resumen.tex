Este proyecto tiene como meta ampliar un prototipo de plataforma existente que se enfoca en el desarrollo de habilidades blandas a través de la gamificación, específicamente para una plataforma gamificada para el aprendizaje de habilidades blandas en ingeniería de sistemas, la cual fue desarrollada simultáneamente con este proyecto por Daniel Felipe Cossio Marulanda. El objetivo de dicha plataforma es mejorar el desempeño académico y laboral de los estudiantes o profesionales de ingeniería de sistemas. La propuesta de este proyecto es integrar dos nuevos módulos específicos: 'Fitness', destinado al bienestar general, y 'Mindset', orientado a transformar las reacciones ante la vida de reactivas a proactivas. Estos módulos buscan complementar y enriquecer la experiencia de aprendizaje dentro del contexto de las habilidades blandas para estudiantes y profesionales de ingeniería de sistemas.
\\ \\
Las habilidades blandas, también llamadas habilidades interpersonales, son esenciales para el éxito en diversos ámbitos. Permiten una comunicación efectiva, la capacidad de resolver problemas, tomar decisiones, fomentar la innovación y liderar proyectos de manera eficaz al interactuar con otros individuos en cualquier entorno laboral o social.
\\ \\
Mientras tanto la gamificación es una táctica educativa que toma elementos característicos de los juegos y los implementa en entornos de aprendizaje y laborales con el propósito de mejorar los resultados. Busca facilitar la asimilación de conocimientos, fomentar el desarrollo de habilidades y premiar logros específicos, entre una variedad de objetivos adicionales. A través de este documento se evidenciará la gamificación para la enseñanza de habilidades blandas, específicamente para las habilidades blandas mindset y fitness, además contendrá los conocimientos necesarios para la comprensión de este proyecto.
\\ \\
\textbf{Palabras clave:} Soft Skills, habilidades blandas, gamificación, módulos.