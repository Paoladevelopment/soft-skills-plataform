%Marco teórico
\section{Marco Teórico}
\begin{itemize}
    \item \textbf{Aprendizaje basado en juegos:} Se analiza teoría de aprendizaje basado en juegos, La cual se refiere a la utilización de juegos como parte integral del procesos educativo, apoyando un entorno más interactivo y significativo en la adquisición del conocimiento.Investigando sus diferentes componentes como son la progresión lineal, la fomentación del aprendizaje, participación activa y la retroalimentación instantánea.
    \\ \\
Esta metodología tiene como finalidad última utilizar juegos con el fin de aprender a través de ellos. El juego se convierte en el vehículo para realizar un aprendizaje o para trabajar un concepto determinado. Mientras dura el juego, o al final de la partida, el docente puede reflexionar en torno a lo que está sucediendo en el juego y los contenidos que se quieren trabajar.\cite{f}

\end{itemize}

\begin{itemize}
    \item \textbf{Gamificación:} En primer lugar conviene expresar que en este proyecto  se utiliza el término gamificación en lugar de ludificación además, se analiza la definición y los principios fundamentales de la gamificación.
    \\ \\
La gamificación es el intento estratégico de mejorar sistemas, servicios, organizaciones y actividades al crear experiencias similares a las experimentadas al jugar juegos, con el fin de motivar e involucrar a los usuarios. Generalmente, esto se logra aplicando elementos de diseño de juegos y principios de juegos (dinámicas y mecánicas) en contextos no relacionados con juegos.\cite{g}
\\ \\
La gamificación forma parte del diseño de sistemas persuasivos y comúnmente utiliza elementos de diseño de juegos para mejorar la participación de los usuarios, la productividad organizacional, el flujo de trabajo, el aprendizaje, la colaboración colectiva, la retención de conocimientos, la contratación y evaluación de empleados, la facilidad de uso, la utilidad de los sistemas, el ejercicio físico, las infracciones de tráfico, la apatía electoral, las actitudes públicas hacia la energía alternativa, entre otros aspectos. Una recopilación de investigaciones sobre gamificación muestra que la mayoría de los estudios encuentran efectos positivos en las personas. Sin embargo, existen diferencias individuales y contextuales
\\ \\
Uno de los principios básicos para la gamificación es el del MDA, uno de los marcos de referencia más prevalentes y empleados en la actualidad para el diseño y evaluación de juegos, se centra en los tres pilares fundamentales de cualquier juego: las mecánicas, las dinámicas y las estéticas.
\\ \\
Las mecánicas de juego se refieren a las reglas que transforman una actividad en una experiencia lúdica o de juego. Estas reglas fomentan la participación y el compromiso de los usuarios al plantear una serie de desafíos y obstáculos que deben superar.
\\ \\
Las dinámicas de juego abarcan los aspectos y valores que moldean la percepción de una actividad. Estos elementos se seleccionan en función del objetivo que se busca alcanzar, como la progresión, la narrativa, la colaboración, entre otros.
\\ \\
La estética del juego se relaciona con el entorno, plataforma o interfaz que permite la experiencia de juego. Se centra en el diseño de la experiencia del usuario, lo cual determina el nivel de atracción que generará en el usuario. Esto abarca tanto los aspectos visuales como táctiles y auditivos de la experiencia de juego.


\end{itemize}
\begin{itemize}
    \item \textbf{Habilidades blandas:}
 Se descomponen e investigan las habilidades blandas también llamadas habilidades sociales o habilidades interpersonales, las cuales afectan la manera como realizamos tareas o interactuamos con los demás, como son la resolución de conflicto,el liderazgo,el trabajo en equipo. 
 \\ \\
Estas capacidades específicas se refieren a habilidades que pueden mejorar el rendimiento laboral, promover la progresión profesional y prever el éxito en el trabajo. También se conocen con diversos nombres como competencias para el siglo XXI, habilidades genéricas, habilidades socioemocionales, entre otros. Estas capacidades abarcan habilidades sociales, interpersonales y metacompetencias que permiten trabajar en entornos diversos y transferir conocimientos de un área a otra.\cite{h} Se examinan estas habilidades, su importancia en el entorno laboral y la vida cotidiana.

\end{itemize}

\begin{itemize}
    \item \textbf{Diseño de experiencia de usuario:}
Se emplea el análisis teórico y práctico del diseño de la experiencia del usuario en relación con la gamificación para desarrollar y aplicar este concepto. Se explorarán aspectos como accesibilidad, usabilidad y diseño de interfaz con el objetivo de ofrecer una experiencia más atractiva y motivadora. 
\end{itemize}

\begin{itemize}
    \item \textbf{Evaluación del aprendizaje:}  Se exploran tácticas y enfoques destinados a medir el progreso de los usuarios en entornos de aprendizaje gamificados. Esto incluye el diseño de métodos que permitan evaluar de manera efectiva tanto el desempeño como el avance en el desarrollo de habilidades. Se enfatiza la importancia de implementar evaluaciones continuas para identificar áreas de mejora, así como la recopilación y análisis de datos relevantes. Estos datos no solo proporcionan una visión integral del progreso, sino que también sirven para personalizar la experiencia de aprendizaje, ajustar las estrategias utilizadas y garantizar que la plataforma cumpla con sus objetivos educativos.

\end{itemize}

\begin{itemize}
    \item \textbf{Teoría de sistemas:} “La Teoría General de Sistemas se concibe como una serie de definiciones, de suposiciones y de proposiciones relacionadas entre sí por medio de las cuales se aprecian todos los fenómenos y los objetos reales como una jerarquía integral de grupos formados por materia y energía;estos grupos son los sistemas.
     \\ \\
Un sistema es un conjunto de fenómenos de objetos con relaciones estrechas entre los unos y los otros y entre los atributos de los mismos. Los sistemas pueden ser de diversidad enorme. La molécula, la célula, el individuo, los grupos sociales, la sociedad y las naciones son todos ejemplos de sistemas vivientes y podrían citarse ejemplos de sistemas no vivientes y muchos también de sistemas combinados que son, por lo demás, los más importantes (sistemas hombre-máquina, por ejemplo).”\cite{i}


\end{itemize}

%Marco conceptual
\section{Marco conceptual}
\begin{itemize}
    
    \item \textbf{Gamificación en la educación:}
    La gamificación ha demostrado su efectividad en la educación, y se han realizado investigaciones que demuestran resultados positivos en el desarrollo de habilidades blandas a través del uso de un enfoque gamificado, como bien lo ha demostrado la revista Ra Ximhai\cite{j} en el caso de la motivación en un aula presencial, sin embargo su uso en  plataformas específicas para el entrenamiento de habilidades blandas es un campo actualmente en crecimiento y que requiere de más investigaciones.
    \item \textbf{Plataformas de e-learning:} Existen un gran número de plataformas de e-learning que buscan el desarrollo de habilidades blandas, pero en el caso de su integración con la gamificación es apenas naciente.estas plataformas a pesar de dar un contenido planificado no se enfocan específicamente en la gamificación para aumentar la motivación y retroalimentación de conocimientos.

    \item \textbf{Tecnologías emergentes:} ”El término TE alude a nuevas tecnologías con potencial de demostrarse como tecnologías disruptivas. Constituyen innovaciones en desarrollo que en un futuro cambiarían la forma de vivir y de producir brindando mayor facilidad a la hora de realizar tareas, o haciéndolas más seguras. Incluyen tecnologías discontinuas derivadas de innovaciones, así como tecnologías más evolucionadas formadas de la convergencia de ramas de investigación antes separadas. Hablar de tecnologías emergentes implica utilizar tecnología para dar soluciones actuales y reales.” \cite{k}
     \\ \\
los avances tecnológicos como la realidad virtual(VR) o realidad aumentada (AR) dan nuevas formas de fomentar el entrenamiento de habilidades blandas mediante la gamificación.Estas tecnologías permiten experiencias de entrenamiento más inmersivas dando hincapié a simulación que puedan aumentar la eficacia de la absorción y aprendizaje de habilidades

    \item \textbf{Enfoque en la personalización y adaptabilidad:} Se ha notado un aumento en la atención hacia la capacidad de personalizar y adaptar las plataformas de aprendizaje, implicando ajustar el contenido,desafíos y actividades de acuerdo a las necesidades de un individuo en particular. lo que fomenta la mejora de la experiencia del usuario e impulsa el desarrollo de habilidades sociales.
    \item \textbf{Retroalimentación:} El análisis de los datos y su uso para una retroalimentación instantánea permite un mejor seguimiento, evaluación y adaptación del desempeño del usuario para mejorar su experiencia así como evaluar sus resultados su impacto en el desarrollo de sus habilidades blandas.
\end{itemize}


%Glosario
\section{Glosario}
\begin{itemize}
    
    \item \textbf{E-learning:} Nos referimos a la formación online a través de dispositivos digitales. Esto implica ver vídeos educativos, leer artículos, realizar cuestionarios e incluso participar en cursos virtuales de e-learning.\cite{l}

    \item \textbf{VR:} Se refiere al grupo de tecnologías usadas para crear una simulación de un entorno o escenas mediante sistemas informáticos.

    \item \textbf{Ingeniería de sistemas:} Disciplina que se enfoca en el diseño, desarrollo, implementación y gestión de sistemas complejos que combinan componentes físicos y software para resolver problemas o satisfacer necesidades.
\item \textbf{Gestión de la información:} Proceso de organización, control y administración de datos para garantizar su accesibilidad, integridad y seguridad.
\item \textbf{Sistemas complejos:} Conjunto de elementos interrelacionados que, en su totalidad, exhiben propiedades y comportamientos que no se pueden entender solo a través de sus partes individuales.
\item \textbf{Interacción efectiva:} Habilidad para comunicarse y colaborar de manera eficiente y positiva con otras personas en diferentes entornos.
\item \textbf{Proactividad:} Actitud de anticipación y toma de iniciativas para resolver problemas o aprovechar oportunidades antes de que surjan.
\item \textbf{Adaptabilidad:} Capacidad para ajustarse y cambiar según las circunstancias, manteniendo el rendimiento y la productividad.
\item \textbf{Inteligencia emocional:} Habilidad para reconocer, comprender y gestionar las emociones propias y de los demás, influyendo en las interacciones personales.
\item \textbf{Trabajo de grado:} Documento académico que descompone un tema específico y constituye una parte integral del proyecto de investigación, presentado como requisito para obtener un título universitario.
\item \textbf{Resiliencia:} Capacidad para adaptarse y recuperarse frente a desafíos o situaciones adversas.
\item \textbf{Aprendizaje activo:} Proceso educativo centrado en la participación activa de los estudiantes en actividades de aprendizaje.


\end{itemize}



%Antecedentes
\section{Antecedentes}
\begin{itemize}
    \item \textbf{El trabajo de grado “Desarrollo de una herramienta de apoyo para el aprendizaje de habilidades blandas con gamificación”:} Escrito por Daniel Felipe Cossio Marulanda se considera el trabajo del que se descompone este.
    \\ \\
Tiene como objetivo buscar la estrategia más adecuada para concebir un prototipo de plataforma “semilla” permita a los estudiantes complementar su formación académica, además de adquirir las habilidades blandas necesarias para el entorno laboral actual, el cual más específicamente se centra en dos habilidades blandas las cuales son el liderazgo y el trabajo en equipo. La plataforma se contruyo mediante el uso de Python y diferentes lenguajes webs, ademas de algunas librerias orientadas al lenguaje de Javascript como es por ejemplo  React.
 \\ \\
El cual aprovecha la metodologia Scrumban la cual es una metodología flexible y adaptable que se puede aplicar a diferentes tipos de proyectos, equipos y contextos, la cual se basa en dos metologias agiles llamdas Scrum y Kanban.

    \item \textbf{El libro “Soft Skills: The Software Developer's Life Manual”:} Escrito por John Sonmez, un desarrollador de software y un popular blogger en el campo de la programación el cual se centra en ayudar a los desarrolladores de software a mejorar sus habilidades no solo desde el punto de vida técnico, sino también en lo que respecta a las habilidades blandas o "soft skills”, las cuales divide en 7 secciones las cuales son carrera,comercializarse a sí mismo,aprendizaje, productividad,finanzas y espíritu.Proporciona consejos prácticos para el crecimiento profesional, la comunicación efectiva, la gestión del tiempo y otros aspectos importantes de la vida de un desarrollador de software.\cite{c}
    
    \item \textbf{El artículo “la gamificación favorece la competencia laboral”:} Se centró en demostrar los resultados de la gamificación como una herramienta de inserción laboral para el estudiantado de la carrera de Administración de la universidad nacional.
    \\ \\
En el cual dicho por los mismos autores “ha permitido comprobar que la gamificación como herramienta de mediación pedagógica incentiva el desarrollo de diferentes habilidades blandas que son necesarias en todos los ambientes en los cuales las personas se desarrollan, pero espacialmente en el mercado laboral.”
\\ \\
Los resultados principales del proceso se centran en la reflexión que surge en los equipos de trabajo, donde el intercambio de ideas y experiencias impulsa el desarrollo de sus habilidades. Además, facilita abordar las debilidades que puedan surgir durante el camino de trabajo, y permite desarrollar habilidades tales como “Adaptación, comunicación, efectiva, resiliencia, toma de decisiones y creatividad”.\cite{m}

    \item \textbf{El artículo "INTEGRACIÓN DE GAMIFICACIÓN Y APRENDIZAJE ACTIVO EN EL AULA":}
    Escrito por Zepeda Hernández Sergio, Abascal Mena Rocío y López  Ornelas Erick utilizó de forma práctica la gamificación en un aula de clase y documentó algunos de sus resultados buscando y un enfoque más activo,entretenido y efectivo para sus alumnos.
    \\ \\
Se documentó la investigación a través de videos en varias clases para analizar el comportamiento, entusiasmo y cambio de actitud de los estudiantes.
\\ \\
como resultado del mismo la interacción grupal mejoró, ya que los estudiantes se ofrecían a ayudarse mutuamente. Se fomenta una actitud más colectiva, donde los estudiantes compartían su conocimiento y ayudaban a sus compañeros.Se observó que la eliminación de exámenes y la evaluación basada en la resolución de actividades y la acumulación de puntos, similar a como sería en un videojuego, generó un mejor ánimo en los estudiantes. Esto incentivó la puntualidad y la asistencia, ya que los estudiantes se dieron cuenta de que cada clase ofrecía la oportunidad de ganar puntos.\cite{j}

    \item \textbf{El artículo "Gamificación en educación: una panorámica sobre el estado de la cuestión":} Escrito por Ana M. Ortiz Colón, Juan Jordán y Míriam Agredal, Se enfoca en revisar el estado actual de la gamificación en el ámbito educativo.El artículo repasa los antecedentes históricos de la gamificación en la educación, incluyendo la aplicación temprana de recompensas y competencias en la enseñanza, así como sus resultados. 
    \\ \\
Además en el mismo se destacan los beneficios de la gamificación en la educación, que incluyen el aumento de la motivación, el compromiso y la retención de conocimientos.Este no limitandose no solo a aspectos positivos sino a varios negativos como desafíos y consideraciones relacionados con la gamificación en la educación. Estos desafíos incluyen la necesidad de un diseño cuidadoso de los juegos, la personalización para adaptarse a diferentes estudiantes y la posibilidad de crear una competencia poco saludable entre los estudiantes.\cite{n}

\end{itemize}

\begin{table}[H]
    \centering
    \begin{tabularx}{\textwidth}{|X|X|}
        \hline
        \rowcolor{naranja} \centering \textbf{Proyecto antecedente} & \multicolumn{1}{|c|}{\textbf{Caracteristicas}} \\ [1mm] \hline
        \centering Desarrollo de una herramienta de apoyo para el aprendizaje de habilidades blandas con gamificación.
        & \begin{itemize}  

\item \textbf Tiene como objetivo el mejorar el desempeño académico y laboral de los estudiantes o profesionales de ingeniería de sistemas
\item \textbf Se centra en las habilidades blandas de liderazgo y trabajo en equipo
\item \textbf Incorpora metodologias de desarrollo de software


\end{itemize} \\
        
        \hline
     
       \centering Soft Skills: The Software Developer’s Life Manual.
        & \begin{itemize}
        \item \textbf Es una guía que se centra en las soft skills que los programadores deben cultivar si desean mejorar su trayectoria profesional
        \item \textbf Incorpora consejos utiles inclusive para campos diferentes a la programacion
        \end{itemize} \\
        \hline
        
        \centering La gamificaciòn favorece la competencia laboral
        & \begin{itemize}
        \item \textbf Reunió al estudiantado de la carrera de administración de la Universidad Nacional para elaborar sus resultados
        \item \textbf En su metodología tomo en cuenta aspectos éticos, así como el uso “INUIT PLACE SLU” la cual es una técnica de aprendizaje en la que la mecánica del juego es aplicada en el campo educativo-profesional y laboral
        \end{itemize}\\
        
        \hline
        \centering Integracion de gamificacion y aprendizaje activo en él aula.
        & \begin{itemize}
        \item \textbf Analizo el comportamiento de grupos de estudiantes de edad temprana
        \item \textbf Logro un resultado satisfactorio en el comportamiento y asistencia de los estudiantes frente a los resultados obtenidos antes de la gamificación
 \end{itemize}\\
        \hline
        
         \centering Gamificación en educación: una panorámica sobre el estado de la cuestión. & \begin{itemize}
        \item \textbf Recopilo información histórica acerca del uso de gamificación en la educación
        \item \textbf Logra recopilar aspectos desafios a tener en cuenta en el uso de la gamificación
        \item \textbf Se enfoca en revisar el estado actual de la gamificaci´on en el ´ambito educativo
\end{itemize}
         \\
         \hline
    \end{tabularx}
    \caption{Tabla sobre proyectos antecedentes}
\end{table}