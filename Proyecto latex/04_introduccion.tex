Para desempeñarse de manera efectiva en la ingeniería de sistemas, se necesita mucho más que conocimientos técnicos o científicos. Las habilidades interpersonales, también conocidas como habilidades blandas, son fundamentales, ya que permiten una interacción fluida con otros individuos, la resolución de problemas, la toma de decisiones, la innovación y el liderazgo en proyectos. Entre las habilidades blandas esenciales para la ingeniería de sistemas se destacan la comunicación efectiva, el trabajo en equipo, la creatividad, el liderazgo y el pensamiento crítico.
\\ \\
En consecuencia es necesario encontrar nuevas estrategias didácticas que promuevan el desarrollo de habilidades blandas en los estudiantes o profesionales de ingeniería de sistemas. Una de ellas es la gamificación, la cual es una estrategia educativa que adopta elementos propios de los juegos en entornos de aprendizaje y trabajo para mejorar los resultados. Su objetivo es facilitar la absorción de conocimientos, el desarrollo de habilidades y la recompensa de logros específicos, entre otros muchos objetivos.\cite{b}
\\ \\
Ante ello, surge la necesidad de crear una plataforma de entrenamiento específica para el desarrollo de habilidades blandas, que brinde a los estudiantes universitarios la oportunidad de adquirir y perfeccionar estas destrezas esenciales para su futuro profesional.
Este proyecto tiene como propósito ampliar un prototipo de plataforma gamificada, diseñado por Daniel Felipe Cossio Marulanda, para el aprendizaje de habilidades blandas en ingeniería de sistemas, el cual fue desarrollado simultáneamente con este proyecto. La plataforma busca mejorar el desempeño académico y profesional de estudiantes y profesionales del área, promoviendo el desarrollo de estas habilidades mediante la gamificación. La ampliación incluye dos nuevos módulos centrados en las áreas de Fitness y Mindset.
\\ \\
El mindset, o mentalidad, refleja la perspectiva personal hacia la vida. Basado en el estoicismo, esta filosofía impulsa a tomar el control de tu existencia en lugar de depender de factores externos. Combinando influencias del estoicismo y el minimalismo, promueve una actitud proactiva en lugar de reactiva, alentando a asumir la responsabilidad y ser el protagonista de tu propia historia.
\\ \\
El fitness, o bienestar físico, es esencial para mantener nuestra salud en óptimas condiciones. En nuestro entorno laboral, pasamos largos periodos de tiempo en posiciones sedentarias, lo que puede acarrear importantes problemas de salud. Esta faceta del bienestar se concentra en aspectos vitales como la nutrición adecuada y la implementación de rutinas de ejercicio personalizadas.\cite{c}
\\ \\
