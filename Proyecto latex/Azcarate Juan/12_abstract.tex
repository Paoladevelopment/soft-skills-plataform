This project aims to expand an existing platform prototype that focuses on the development of soft skills through gamification, more specifically for a gamified platform for learning soft skills for systems engineering, which was developed simultaneously with this project by Daniel Felipe Cossio Marulanda. The objective of this platform is to improve the academic and work performance of systems engineering students or professionals. The proposal for this project is to integrate two new specific modules: 'Fitness', aimed at general well-being, and 'Mindset', aimed at transforming reactions to life from reactive to proactive. These modules seek to complement and enrich the learning experience within the context of soft skills for systems engineering students and professionals.
\\ \\
Soft skills, also known as interpersonal skills, are essential for success in various fields. They enable effective communication, problem-solving abilities, decision-making, fostering innovation, and leading projects effectively by interacting with others in any work or social environment.
\\ \\
Meanwhile, gamification is an educational tactic that adopts characteristic elements of games and applies them in learning and work environments to enhance outcomes. It aims to facilitate knowledge assimilation, promote skill development, and reward specific achievements, among a variety of additional objectives. This document will showcase gamification for teaching soft skills, specifically for the mindset and fitness soft skills, and will also contain the necessary knowledge for understanding this project.
\\ \\
\textbf{Keywords:} Soft Skills, gamification, modules.

