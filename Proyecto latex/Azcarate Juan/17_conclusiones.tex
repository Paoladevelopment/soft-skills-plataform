\section{Conclusiones}

Las habilidades blandas han demostrado ser un tema extenso y esencial cuya adquisición representa una ventaja significativa para los estudiantes. En el ámbito del desarrollo de software, donde los proyectos enfrentan constantes cambios, es crucial un enfoque adaptable. Durante la elaboración de este documento, se observó que las personas abordan los desafíos de formas muy diversas: algunas prefieren la repetición constante, mientras que otras tienden a reflexionar profundamente antes de actuar. Sin embargo, todas ellas muestran una capacidad notable para adaptarse y, mediante la perseverancia, logran alcanzar resultados satisfactorios, incluso cuando el proceso requiere tiempo.
\\ \\
El desarrollo de los módulos de mindset y fitness para una plataforma educativa gamificada permitió evidenciar el impacto positivo de la gamificación en el fortalecimiento de habilidades blandas esenciales para estudiantes y profesionales de ingeniería de sistemas. A través de técnicas innovadoras, como retos interactivos y dinámicas de juego, se logró fomentar la participación activa y crear un entorno propicio para el aprendizaje continuo y la mejora personal.
\\ \\
Los resultados obtenidos durante las pruebas revelaron observaciones interesantes, como la preferencia de los estudiantes por enfoques visuales y llamativos. En este sentido, los módulos que implicaron mayor movilidad física, como el de ejercicios para la espalda, lograron captar más la atención de los participantes y motivarlos a continuar jugando. Además, estas actividades promovieron prácticas beneficiosas tanto para la vida laboral como cotidiana, ofreciendo conocimientos prácticos para prevenir problemas corporales futuros y fomentando habilidades analíticas mediante la identificación de patrones.
\\ \\
Asimismo, los resultados indican que estos módulos cumplieron con los objetivos planteados: promover el bienestar físico y mental mientras fortalecen competencias transversales como la resiliencia, el pensamiento crítico y la capacidad de adaptación. Durante las pruebas, se destacó el papel de la memoria en la resolución de problemas, demostrando que las estrategias cognitivas individuales influyen significativamente en el desempeño.
\\ \\
En conclusión, este proyecto no solo contribuye a la formación integral de los estudiantes de ingeniería de sistemas, sino que también enriquece el campo de la educación en habilidades blandas, ofreciendo un modelo replicable y adaptable a otras disciplinas, consolidando así su valor académico y profesional.

\section{Trabajos futuros}

\begin{enumerate}
    \item El módulo fitness sobre movimiento de la espalda requiere una barra de estado o cuenta regresiva que indique al usuario cuánto tiempo queda de la canción. Esto mejoraría la percepción del progreso y la motivación del jugador.
    
    \item Aunque las instrucciones de los módulos fitness fueron satisfactorias, se observó que el módulo mindset carece de una guía explícita. Por lo tanto, sería recomendable implementar un sistema de instrucciones para mejorar la comprensión del usuario.
    
    \item Durante las pruebas, el módulo mindset presentó problemas de rendimiento al ejecutarse en React. Por esta razón, se optó por utilizar un servidor externo para mejorar la experiencia. Sin embargo, se recomienda desarrollar una versión optimizada dentro de la misma aplicación para personalizar aún más la interacción del usuario con el módulo.
    
    \item Observaciones realizadas sobre las instrucciones del módulo fitness para la espalda sugieren que estas podrían ser insuficientes. Por ello, se recomienda ampliarlas para asegurar que los usuarios comprendan plenamente las dinámicas del juego.
    
    \item Dado que este proyecto es un prototipo, se desarrollaron pocos niveles y se contó con un número limitado de usuarios para las pruebas. Se recomienda ampliar la cantidad de niveles y realizar evaluaciones con un mayor número de participantes, incluyendo estudiantes de diferentes carreras para diversificar los resultados.
    
    \item Debido a la cantidad de movimientos y gestos medidos en los elementos gamificados, el rendimiento puede verse afectado en equipos de cómputo con menor capacidad. Sería factible optimizar el código para mejorar la experiencia del usuario en equipos con especificaciones más modestas.
\end{enumerate}