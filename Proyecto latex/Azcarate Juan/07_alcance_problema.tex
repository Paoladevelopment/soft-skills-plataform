%Declaración del alcance
\section{Declaración del alcance}
El objetivo principal de este proyecto es el diseño y gamificación de los módulos de fitness y mindset a un prototipo de plataforma educativa enfocada en el desarrollo de habilidades blandas específicamente diseñadas para estudiantes y profesionales de ingeniería de sistemas. La meta es generar un impacto positivo tanto en el desempeño académico como en la preparación laboral de los involucrados en esta disciplina. A su vez, se busca ampliar el entendimiento sobre la efectividad de la gamificación en la enseñanza y el aprendizaje, proporcionando así una visión más amplia de su aplicación.

\section{Supuestos}
\begin{itemize}
\item \textbf
Se cuenta con la plataforma prototipo de plataforma educativa enfocada en el desarrollo de habilidades blandas.
\item \textbf
Los módulos se desarrollarán como un prototipo funcional abierto a futuras integraciones y mejoras.
\item \textbf
Existen fuentes confiables y actualizadas de información sobre las habilidades blandas requeridas por las empresas y las áreas de mejora más comunes.
\item \textbf
Los estudiantes de ingeniería de sistemas tienen acceso a internet y a dispositivos móviles o computadores para usar la plataforma.
\item \textbf
Los estudiantes de ingeniería de sistemas están motivados e interesados en aprender y entrenar habilidades blandas.
\end{itemize}
\section{Restricciones}
\begin{itemize}
\item \textbf
La plataforma debe respetar los derechos de autor y las normas éticas al usar contenido educativo de otras fuentes.
\item \textbf
Los módulos deben de ser desarrollados en un lenguaje de programación y una herramienta de diseño que sean compatibles con los recursos disponibles.
\item \textbf
Se abordarán 2 habilidades blandas .
\item \textbf
Los prototipos de los módulos deben ser desarrollados en un periodo de tiempo de 8 meses como máximo.

\end{itemize}
%Objetivos
\section{Objetivos}

\subsection{Objetivo general}
Desarrollar y gamificar los módulos de mindset y fitness para el prototipo de plataforma enfocada en la enseñanza de habilidades blandas.

\subsection{Objetivos especificos}
\begin{enumerate}
    \item Revisar la literatura enfocada en analizar las habilidades blandas mindset y fitness.
    
    \item Desarrollar los módulos de mindset y fitness.
    
    \item Integrar características de gamificación en los módulos de mindset y fitness.
    
    \item Evaluar la usabilidad de los módulos.
    
\end{enumerate}
\section{Entregables esperados}
\begin{itemize}
\item \textbf
Código fuente de los módulos y especificaciones previamente definidos.
\item \textbf
Un documento que contenga pruebas de funcionamiento y validación con usuarios potenciales.
\item \textbf
Un documento que contenga el diseño y la especificación de los módulos, incluyendo las características de gamificación, el contenido educativo, las actividades prácticas y los criterios de evaluación
\item \textbf
Un documento que contenga la revisión de literatura sobre las habilidades blandas seleccionadas.
\end{itemize}

%Resultados esperados
\section{Resultados esperados}

\begin{table}[H]
    \centering
    \begin{tabularx}{\textwidth}{|X|X|}
        \hline
        \rowcolor{naranja} \centering \textbf{Objetivo Específico} & \multicolumn{1}{|c|}{\textbf{Resultado Esperado}} \\ [1mm] \hline
        Revisar la literatura enfocada en analizar las habilidades blandas mindset y fitness.
        & Un informe de la literatura recolectada, enfocándose en destacar las habilidades blandas mindset y fitness, identificando las razones por las cuales estas habilidades son altamente demandadas por las empresas en el ámbito laboral, así como su utilidad para el desarrollo del educando. \\
        \hline
        Desarrollar los módulos de mindset y fitness.
        & El código fuente de los módulos mindset y fitness. \\
        \hline
        Integrar características de gamificación en los módulos de mindset y fitness.
        & El código fuente  fuente de los módulos mindset y fitness con la funcionalidad de gamificación completamente implementada y probada. \\
        \hline
        Evaluar la usabilidad de los módulos en la plataforma.
        & Un informe que presente los resultados de la evaluación del impacto de la gamificación en la experiencia de los usuarios, con conclusiones y recomendaciones para futuras mejoras. \\
        \hline
    \end{tabularx}
    \caption{Resultados Esperados}
\end{table}

